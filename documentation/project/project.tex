\chapter{Projektmanagement}

\section{Aufgabe}

Die Aufgabe bestand darin, einen funktionsfähigen Prozessor auf Basis der RISC-V Instruction Set Architecture zu entwickeln, der auf dem FPGA des gegebenen Entwicklungsboards (Spartan-3A FPGA Starter Kit) läuft. 

(Bild vom Board (evtl. mitsamt Monitor))


\section{Basisziele (Pflichtangebot)}
Als Mindestanforderung sollte ein Prozessor mit Kompatibilität zur „RV32I Base Integer Instruction Set“ implementiert werden. Ausgenommen davon sind die Befehle FENCE, FENCE.I, SCALL und SBREAK, da keine Hardwareunterstützung für den Multitaskingbetrieb benötigt wird.
Um die Funktionsfähigkeit des Prozessors auch nach außen sichtbar zu machen und keine Black-Box zu erstellen soll die Möglichkeit der Interaktion über die Schnittstellen und Pins des Boards bestehen. Insbesondere soll zum Debugging eine grafische Registerausgabe über den VGA-Port an einem Monitor möglich sein.

Siehe Anlage (Abgegebenes)


\section{Erweiterungsziele (Erweitwerungsangebot)}
Aufgrund der Modularität von RISC-V bietet es sich an mindestens eine Erweiterung zu implementieren, nämlich die „RV32M Standard Extension for Integer Multiplication and Division“, die Multiplikations- und Divisionsbefehle beinhaltet. Zudem soll die rudimentäre Ausgabe um einen Textmodus erweitert werden, sodass mittels Memory-Mapping ASCII-Zeichen auf dem Monitor ausgegeben werden können. Zum Demonstrieren der Funktionalität soll außerdem ein auf dem Prozessor lauffähiges Spiel entwickelt werden. Um auch Datenübertragung mit der Außenwelt zu ermöglichen soll die serielle Schnittstelle genutzt werden.

\section{Verwendete Tools}
Das Projekt wurde in VHDL implementiert, da sämtliche Programmierer ausschließlich in dieser Hardwareprogrammierungssprache gute Kenntnisse hatten.
Zur Entwicklung wurden hauptsächlich Xilinx‘ „ISE Project Navigator“ verwendet. Dieser diente zugleich als Editor für den VHDL-Code und auch als Werkzeug um daraus die Programming-Files, mit denen das FPGA beschrieben wird zu generieren. Der integrierte „Core Generator“ wurde benutzt um einzelne Module zu erstellen.
Mittels Xilinx‘ „impact“ in Kombination mit dem „Cable-Server“ wurde das Board über USB gemäß den Programming-Files beschrieben.
Zur Versionsverwaltung wurde auf ein Github Repository gesetzt.
Um nicht jedes zu testende Programm per Hand assemblieren zu müssen wurde ein Assemblierer auf Python-Basis erstellt.
Da die Generierung eines Programming-Files mit anschließendem Beschreiben des FPGAs mehrere Minuten in Anspruch nimmt wurden zum Testen Simulatoren verwendet. Um den VHDL-Code zu verifizieren wurde so auf GHDL in Kombination mit GTKWave gesetzt. Auch zur Assemblerprogrammierung wurde ein Simulator mitsamt Textausgabe entwickelt um schneller debuggen zu können.


