

\chapter{Die ASCII-Unit}
Die ASCII-Unit ist für die Textausgabe auf dem Monitor zuständig. 
\section{Überblick}
Die ASCII-Unit gibt auf dem Monitor mittels Memory-Mapping den Inhalt des CHARRAMs gemäß ASCII-Kodierung aus. Dazu wird den Zeichen-Stellen auf dem Monitor jeweils eine Adresse zugeordnet: Die Stelle im Eck oben links erhält die 0, nach rechts wird bis zum Zeilenende durchinkrementiert, dann wird jeweils in der nächsten Zeile fortgefahren, sodass sich insgesamt 32 Zeilen à 64 Zeichen ergeben und die letzte Stelle rechts unten die Adresse 2047 erhält. Diese Adressen beziehen sich genau auf die 2048 Byte des CHARRAMs in der MMU, die eben jeweils genau ein Zeichen repräsentieren.
Um jedem Zeichen neben seiner ASCII-Nummer auch das entsprechende Aussehen zuzuordnen zu können wurde auf eine CHARMAP gesetzt, die zu jedem der 256 ASCII Zeichen einen 64-Bit-Vektor eingespeichert hat, der eben das 8*8 Bitfeld eines jeden Zeichens repräsentiert. Von diesen 8*8 Bit sind jeweils nur 6*6 für das Zeichen reserviert, die Restlichen sind immer ungesetzt, sodass auf dem Monitor ein Abstand von 2 freien Pixeln zwischen zwei nebeneinanderliegenden Zeichen bleibt. Diese Entscheidung auf Kosten der Speichervergeudung ermöglichte eine einfachere Implementierung, da für die Berechnungen ASCII-Unit Divisionen und Multiplikationen notwendig sind und diese mit Zweierpotenzen (hier 8,32 bzw. 64) erheblich einfacher zu realisieren sind. 
Da eine Darstellung von Kleinbuchstaben auf einem 6*6 Bitfeld kaum sinnvoll zu realisieren ist, eine Unterscheidung ohnehin schwierig wäre und der Fokus nicht auf Ästhetik lag wurden diese durch die Großbuchstaben ersetzt.
Nicht zeichenbare ASCII-Zeichen (Zeilenumbrüche, Tabulator etc.) werden als leere Zeichen dargestellt, sodass bei der Berechnung der Adresse nicht auf vorangegangene Sonderzeichen geachtet werden muss, was die Implementierung erleichtert. Dadurch muss die Software die Verwaltung des CHARRAMS übernehmen, um diese Zeichen korrekt darzustellen.
Die ASCII-Unit ist wie die VGA-Unit auf 25 MHz getaktet. Denn die VGA-Unit sendet in jedem ihrer Takte die Informationen zu einem Pixel zum Monitor und die ASCII-Unit berechnet in jedem Takt genau die Information, ob das jeweilige Pixel gesetzt oder frei, sprich Weiß oder Schwarz sein soll. Die Berechnung des Pixels in der ASCII-Unit erfolgt dabei über mehrere Takte hinweg stufenweise.


Schaubild ZeichenGitter+Adrressen


\section{Interface}
Neben dem bereits erwähnten Takteingang gibt es jeweils einen Eingang für die x- und y- Koordinate des Fadenstrahls, sprich des aktuellen Pixels, aus der VGA-Unit, welcher sich mit jedem Takt ändert und ein Ausgangssignal an die VGA-Einheit, welches angibt, ob das aktuelle Pixel gesetzt werden soll oder nicht. Außerdem gibt es zur Kommunikation mit dem CHARRAM in der MMU einen Ausgang, welcher die Adresse des aktuell zu berechnenden Pixels angibt sowie einen Eingang, der im darauffolgenden Takt die ASCII-Nummer des Zeichens erhält.
Zur CHARMAP wird der Takt durchgeleitet und ebenso die aus dem CHARRAM kommende Adresse des aktuellen Zeichens. Aus der CHARMAP kommt im darauffolgendem Takt der oben erwähnte 64-Bit-Vektor der das jeweilige Zeichen repräsentiert.

\section{Funktionsweise}
Die stufenweise Berechnung für jedes Pixel beginnt mit der Verrechnung der x- und y- Koordinate aus der VGA-Einheit. Dazu wird zuerst die x-Koordinate um 2 erhöht und in ein internes Signal gespeichert, sodass quasi im Vorraus berechnet wird.
Im nächsten Takt wird daraus die Adresse des aktuellen Zeichen-Platzes berechnet, welche durch die MMU in den CHARRAM geleitet wird.  round_down(y/8)*64+ round_down(x_mod/8)
Von dort wird im nächsten Takt die ASCII-Nummer des aktuellen Zeichens geliefert, welche direkt in die CHARMAP weitergeleitet wird. Dies ist auch problemlos bei gleichzeitigem Zugriff der MMU auf den CHARRAM möglich, da dieser als Dual-Port-Blockram realisiert wird. Lediglich falls zwischen den insgesamt 64 Zugriffen pro Frame der ASCII-Unit auf eine Adresse im CHARRAM eben diese Zelle von der MMU überschrieben wird kann es zu einer kurzzeitigen Störung kommen: Dieses Zeichen würde dann möglicherweise in diesem Frame falsch dargestellt, was sich aber im Betrieb allerdings kaum bemerkbar macht.
Die CHARMAP liefert im nächsten Takt wiederum den aktuellen 64-Bitvektor aus dem mittels der aktuellen x- und y Koordinate berechnet wird, ob das Pixel zu setzen ist oder nicht. Dazu …

Insgesamt benötigt die Berechnung also … Takte
