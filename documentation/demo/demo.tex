\chapter{Entwicklung eines Demo-Programms}

Das folgende Kapitel erläutert die Struktur, Motivation und Vorgehensweise hinter dem zu demonstrativen Zwecken entwickelten, auf dem Prozessor lauffähigen Tic-Tac-Toe Programms. Dieses vereint alle zentralen Funktionalitäten der Prozessor-Implementierung auf dem FPGA.

\section{Struktur und Funktion des Demo-Programms}

Wie eingangs geschildert bietet das entwickelte Demo-Programm die Möglichkeit, über die Buttons des FPGAs gegeneinander Tic-Tac-Toe zu spielen. Dabei wird mittels der Buttons BTN 0 bis BTN 3 (siehe Kapitel \ref{sec:mmuio} das Steuerkreuz über die das Spielbrett navigiert. Dabei ist das Brett zyklisch angeordnet, sodass eine Linksbewegung des Steuerkreuzes an den linken Rand dieses an die rechte Brettseite manövriert. Mittels des Buttons BTN 4 (siehe Kapitel)\ref{sec:mmuio} setzt der Zugspieler seine Markierung an der zuvor ausgewählten Position, sofern dies regelkonform ist. Das Spiel enthält keinen internen Reset und muss daher über einen Hard-Reset durch das FPGA erfolgen. 

\subsection{Umsetzung}

Um möglichst viel der implementierten Funktionalität abzudecken, wurde bei der Implementierung darauf geachtet, die meisten Komponenten zu beanspruchen. So wird bei Programmstart im DDR2-SDRAM-Block eine als Array von vorzeichenbehafteten 8-Bit Werten realisierte Matrix $M$ wie folgt initialisiert.

$ M = \begin{pmatrix} -4 & -4 & -4\\ -4 & -4 & -4\\ -4 & -4 & -4  \end{pmatrix} $

Dabei ist der Eintrag wie folgt definiert:
