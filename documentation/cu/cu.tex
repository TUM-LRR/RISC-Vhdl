
\Chapter{Das Leitwerk}
\label{ch:cu}

\Section{\"Uberblick}
Das Leitwerk \textit{CU} stellt die zentrale Steuereinheit des Prozessors dar. Es
interpretiert die Befehle und \"uberwacht ihre Ausf\"uhrung durch ALU und MMU.
Der Fokus lag dabei mehr auf Robustheit und weniger auf Aspekten der
Performanz. Au\ss{}erdem verwaltet das Leitwerk den Program-Counter~(PC) und
das Instruction-Register~(IR).

Es besitzt f\"ur jeden Befehl eine eigene Zustandsmaschine, deren Verhalten 
durch den Inhalt des Instruction-Register gesteuert wird. Daher muss jeder
Befehl als letztes das IR mit dem Opcode des folgenden Befehl laden. Wann dies
geschieht kann in jedem Befehl einzeln optimiert werden.

Um den Implementierungsaufwand bei \"Anderungen von Befehlen zu minimieren
wurde ein Compiler-Skript erstellt das mehrere Makros bereitstellt, aus denen
dann die Befehle zusammengebaut werden k\"onnen. Das Skript kompiliert dann
eine Eingabe aus diesen Makros in VHDL-Code.

\Subsection{Legende}
Da jeder Befehl eine eigene Zustandsmaschine besitzt, wird hier f\"ur jeden
Befehl ein eigenes Zustands\-\"uber\-gangs\-dia\-gramm gezeigt. Die Pr\"afixe
``MMU:'' und ``ALU:'' werden genutzt, um anzuzeigen, dass eine Operation an die
jeweiligen Einheit delegiert wird und das Leitwerk lediglich eine Anfrage an
die entsprechende Komponente sendet.

Obwohl in der ISA regelm\"a\ss{}ig verlangt wird, dass Operanden sign-extended
werden, wurde dies in den Zustands\-\"uber\-gangs\-dia\-grammen der \"Ubersicht
halber weggelassen. Im Prozessor ist es selbstverst\"andlich wie in der ISA
beschrieben implementiert.

\Section{Integer Rechenbefehle}
Der Prozessor wurde auf die in \textit{RV32I} und \textit{RV32M} definierten Rechenbefehle
optimiert. Dadurch k\"onnen diese RISC-typischen Befehle sehr schnell
ausgef\"uhrt werden.

\Statemachine{./cu/img/integer_rechenbefehle.pdf}{
Zu\-stands\-\"uber\-gangs\-dia\-gramm der Befehle \Instr{AND[I]},
\Instr{OR[I]}, \Instr{XOR[I]}, \Instr{SLL[I]}, \Instr{SRL[I]}, \Instr{SRA[I]},
\Instr{ADD[I]}, \Instr{SUB}, \Instr{SLT[I][U]}, \Instr{MUL} und
\Instr{MULH[[S]U]}.}{\(op2\) ist entweder das Register, das mit \Ipart{rs2}
angegeben ist, oder eine Immediate (\Ipart{imm}). ``\(\circ\)'' repr\"asentiert
die jeweilige Operation (\(+\), \(-\), \dots{}).
}

\Subsection{Division und Modulo}
Da bei der Division unm\"oglich zu garantieren ist, dass diese immer innerhalb
von drei Takten erfolgreich beendet wird, muss das Leitwerk hier auf eine
Best\"atigung der ALU\ref{ch:alu} warten. Diese sieht vor, dass das Rechenwerk
die Leitungen \Vhdl{alu\_data\_in} auf 0 setzt.

\Statemachine{./cu/img/division.pdf}{
Zustands\"ubergangsdiagramm der Befehle \Instr{DIV[U]} und \Instr{REM[U]}.
}{``\(\circ\)'' repr\"asentiert die jeweilige Operation (\(/\), \(\bmod\),
\dots{}).}

\Section{LUI und AUPIC}
Da durch die Integer Rechenbefehle keine 32-Bit Immediates direkt geladen werden
k\"onnen, definiert die RISC-V-ISA die Befehle \Instr{LUI} und \Instr{AUPIC}, die
diesen Mangel beheben.

\Statemachine{./cu/img/lui_und_aupic.pdf}{
Zustands\"ubergangsdiagramm der Befehle \Instr{LUI} und \Instr{AUPIC}.
}{\(op2\) ist entweder 0 (bei \Instr{LUI}) oder der aktuelle Program-Counter
(bei \Instr{AUPIC}).}


\Section{Bedingte Spr\"unge}
Da bei Speicherzugriffen auf den DDR2-SDRAM Block der MMU\ref{ch:mmu} nicht
garantiert werden kann, dass ein Speicherzugriff ohne Zeit- und Datenverlust
abgebrochen werden kann, wurde auf Branch-Prediction g\"anzlich verzichtet.
Dadurch geh\"oren die bedingten Spr\"unge im Hinblick auf Rechenzeit zu den
teuersten Befehlen des Prozessors.

\Statemachine{./cu/img/beq_und_bgeu.pdf}{
Zustands\"ubergangsdiagramm der Befehle \Instr{BEQ} und \Instr{BGE[U]}.}
{``\(\circ\)'' ist bei \Instr{BEQ} ``\(-\)'', bei \Instr{BGE} die SLT-Operation
und bei \Instr{BGEU} die SLTU-Operation.
}

\Statemachine{./cu/img/bne_und_bltu.pdf}{
Zustands\"ubergangsdiagramm der Befehle \Instr{BNE} und \Instr{BLT[U]}.}
{``\(\circ\)'' ist bei \Instr{BNE} ``\(-\)'', bei \Instr{BLT} die SLT-Operation
und bei \Instr{BLTU} die SLTU-Operation.
}

\Section{Unbedingte Spr\"unge}
Anders als bei bedingten Spr\"ungen, kann bei unbedingten Spr\"ungen das
Sprungzeil immer vorhergesagt werden. Dadurch kann das Schreiben der
Return-Adresse und das Holen des n\"achsten Befehls parallelisiert werden, was
zu einer merklichen Geschwindigkeitssteigerung f\"uhrt.

\Statemachine{./cu/img/unbedingte_spruenge.pdf}{
Zustands\"ubergangsdiagramm der Befehle \Instr{JAL} und \Instr{JALR}.}
{\(op2\) ist bei \Instr{JAL} der Program-Counter und bei \Instr{JALR} das
Register, welches durch \Ipart{rs1} adressiert wird.
}

\Section{LOAD}
Der LOAD-Befehl l\"adt aus dem Speicher immer einen 32-Bit Wert, den das
Leitwerk dann zuschneidet. Dies sollte urspr\"unglich die Implementierung der
MMU vereinfachen und aligned-Speicherzugriffe beschleunigen. Es hat sich
allerdings herausgestellt, dass diese Entscheidung derzeit nur Nachteile mit
sich bringt. Die Ausf\"uhrungsdauer des Befehls wird in der Praxis
haupts\"achlich von der adressierten Speicherkomponente bestimmt.

\Statemachine{./cu/img/load.pdf}{
Zustands\"ubergangsdiagramm der Befehle \Instr{LB[U]}, \Instr{LH[U]} und
\Instr{LW}.}{\(width\) ist je nach Befehl 1,~2~oder~4.
}

\Section{STORE}
Der Store-Befehl ist mit nur einem Rechen- und Speicherwerk nicht zu
parallelisieren, was dazu f\"uhrt, dass er den langsamsten Befehl des
Leitwerks darstellt. Da Schreibzugriffe jedoch generell auf RISC-Architekturen
sehr hohe Kosten hinsichtlich der ben\"otigten Rechenzeit aufweisen, sind
Compiler und Programmierer ohnehin dazu angehalten, schreibende
Speicherzugriffe m\"oglichst zu vermeiden.

\Statemachine{./cu/img/store.pdf}{
Zustands\"ubergangsdiagramm des Befehle \Instr{SB}, \Instr{SH} und \Instr{SW}.}
{\(width\) ist je nach Befehl 1, 2 oder 4.
}

\Section{Timer und Counter}
Wie in der RISC-V-ISA gefordert existiert des Weiteren ein Counter, der die
Anzahl der bisher ausgef\"uhrten Befehle z\"ahlt. Au\ss{}erdem gibt es einen
Timer, der analog dazu die Anzahl der bereits vergangenen Takte speichert. Da
das Entwicklungsboard allerdings keine Echtzeituhr bereitstellt, greift die
Implementierung dieser Funktionalit\"at auf den Taktz\"ahler zur\"uck.

Ausgelesen werden k\"onnen diese Counter durch die Befehle
\Instr{RDINSTRET[H]}\nolinebreak{}, \Instr{RDCYCLE[H]} und \Instr{RDTIME[H]}.

\Statemachine{./cu/img/timer_und_counter.pdf}{
Zustands\"ubergangsdiagramm der Befehle \Instr{RDINSTRET[H]},
\Instr{RDCYCLE[H]} und \Instr{RDTIME[H]}.}{\(counter\) sind die oberen bzw.
unteren 32 Bit des jeweiligen 64 Bit Z\"ahlers.
}
