\Chapter{Das Leitwerk}
Das Leitwerk ist die zentrale Steuereinheit des Prozessors. Es interpretiert
die Befehle und \"uberwacht ihre Ausf\"uhrung durch ALU und MMU. Dazu verwaltet
das Leitwerk den Program-Counter (PC) und das Instruction-Register (IR).

\Section{\"Uberblick}
Im Prozessor wurden die durch das \ISAname{RV32I Base Integer Instruction Set}
und die \ISAname{RV32M Standard Extension for Integer Multiplication and
Division} definierten Befehle implementiert. Eine Ausnahme bilden dabei alle
Befehle, die Multitasking erm\"oglichen sollen, also \Instr{SCALL}
\nolinebreak{}, \Instr{SBREAK}\nolinebreak{}, \Instr{FENCE} und
\Instr{FENCE.I}. Besondere Aufmerksamkeit wurde dabei mehr auf Robustheit und
weniger auf maximale Geschwindigkeit gelegt.

Das Leitwerk besitzt f\"ur jeden Befehl eine eigene Zustandsmaschine. Welche
ausgef\"uhrt wird h\"angt allein vom Inhalt des Instruction-Register ab. Daher
muss jeder Befehl als letztes das IR mit dem folgenden Befehl belegen. Wann
dieser Befehl geladen wird kann nun in jedem Befehl einzeln optimiert werden.

Um den Implementierungsaufwand bei \"Anderungen von Befehlen zu minimieren
wurde ein Compiler-Skript erstellt, das mehrere Makros bereitstellt, aus denen
dann die Befehle zusammengebaut werden k\"onnen. Das Skript kompiliert dann
eine Eingabe aus diesen Makros in VHDL-Code.

\Subsection{Legende}
Da jeder Befehl eine eigene Zustandsmaschine besitzt wird hier f\"ur jeden
Befehl ein eigenes Zustands\-\"uber\-gangs\-dia\-gramm gezeigt. Die Pr\"afixe
"`MMU:'' und "`ALU:'' werden genutzt um anzuzeigen, dass eine Aktion von der
jeweiligen Einheit ausgef\"uhrt wird und das Leitwerk lediglich eine Anweisung
gibt.

In der ISA wird regelm\"a\ss{}ig verlangt, dass Operanden sign-extended werden.
Dies wird in den Zustands\-\"uber\-gangs\-dia\-grammen der \"Ubersicht halber
weggelassen. Im Prozessor ist es selbstverst\"andlich wie in der ISA
beschrieben implementiert.

\Section{Integer Rechenbefehle}
Der Prozessor wurde auf die in RV32I und RV32M definierten Rechenbefehle
optimiert. Dadurch k\"onnen diese RISC-typischen Befehle sehr schnell
ausgef\"uhrt werden.

\Statemachine{./cu/img/integer_rechenbefehle.pdf}{
Zustands\"ubergangsdiagramm der Befehle \Instr{ADD[I]}, \Instr{SUB},
\Instr{SLT[I][U]}, \Instr{AND[I]}, \Instr{OR[I]}, \Instr{XOR[I]},
\Instr{SLL[I]}, \Instr{SRL[I]}, \Instr{SRA[I]}, \Instr{MUL[W]} und
\Instr{MULH[[S]U]}. \(op2\) ist entweder das Register, das mit \Ipart{rs2}
angegeben ist, oder eine Immediate (\Ipart{imm}). ``\(\circ\)'' repr\"asentiert die
jeweilige Operation (\(+\), \(-\), \dots{}).
}

\Subsection{Division und Modulo}
Da bei der Division unm\"oglich zu garantieren ist, dass diese immer nach drei
Takten beendet ist, muss das Leitwerk hier auf eine Best\"atigung der ALU
warten. Diese sieht vor, dass das Rechenwerk die Leitungen \Vhdl{alu\_data\_in}
auf 0 setzt.

\Statemachine{./cu/img/division.pdf}{
Zustands\"ubergangsdiagramm der Befehle \Instr{DIV[U][W]} und \Instr{REM[U][W]}.
``\(\circ\)'' repr\"asentiert die jeweilige Operation (\(/\), \(\bmod\), \dots{}).
}

\Section{LUI und AUPIC}
Da durch die Integer Rechenbefehle keine 32-Bit Immediates direkt geladen werden
k\"onnen, definiert die RISC-V-ISA die Befehle \Instr{LUI} und \Instr{AUPIC}, die
diesen Mangel beheben. Auch diese Befehle wurden auf Geschwindigkeit optimiert.

\Statemachine{./cu/img/lui_und_aupic.pdf}{
Zustands\"ubergangsdiagramm der Befehle \Instr{LUI} und \Instr{AUPIC}.
\(op2\) ist entweder 0 (bei \Instr{LUI}) oder der aktuelle Program-Counter
(bei \Instr{AUPIC}.
}

\Section{Bedingte Spr\"unge}
Da bei dem DDR2-Speicher des benutzten Boards nicht garantiert werden
konnte, dass ein Speicherzugriff ohne Zeit- und Datenverlust abgebrochen
werden kann, wurde auf eine einfache Branch-Prediction g\"anzlich verzichtet.
Dadurch sind die bedingten Spr\"unge unter den teuersten Befehlen des
Prozessors.

\Statemachine{./cu/img/beq_und_bgeu.pdf}{
Zustands\"ubergangsdiagramm der Befehle \Instr{BEQ} und \Instr{BGE[U]}.
``\(\circ\)'' ist bei \Instr{BEQ} ``\(-\)'', bei \Instr{BGE} die SLT-Operation
und bei \Instr{BGEU} die SLTU-Operation.
}

\Statemachine{./cu/img/bne_und_bltu.pdf}{
Zustands\"ubergangsdiagramm der Befehle \Instr{BNE} und \Instr{BLT[U]}.             
``\(\circ\)'' ist bei \Instr{BNE} ``\(-\)'', bei \Instr{BLT} die SLT-Operation
und bei \Instr{BLTU} die SLTU-Operation.
}

\Section{Unbedingte Spr\"unge}
Anders als bei bedingten Spr\"ungen, kann bei unbedingten Spr\"ungen das
Sprungzeil immer vorhergesagt werden, wodurch das schreiben der Return-Adresse
und das holen des n\"achsten Befehls parallelisiert werden kann, was zu einer
merklichen Geschwindigkeitssteigerung f\"uhrt.

\Statemachine{./cu/img/unbedingte_spruenge.pdf}{
Zustands\"ubergangsdiagramm der Befehle \Instr{JAL} und \Instr{JALR}. \(op2\)
ist bei \Instr{JAL} der Program-Counter und bei \Instr{JALR} 
}

\Section{LOAD}
Der LOAD-Befehl l\"adt aus dem Speicher immer einen 32-Bit Wert, den das
Leitwerk dann zuschneidet. Dies sollte urspr\"unglich die Implementierung der
MMU vereinfachen und aligned-Speicherzugriffe beschleunigen, es hat sich
allerdings herausgestellt, dass diese Entscheidung derzeit nur Nachteile mit
sich bringt. Die Ausf\"uhrung des Befehls ist durchschnittlich und wird in der
Praxis haupts\"achlich von der Speichergeschwindigkeit bestimmt.

\Statemachine{./cu/img/load.pdf}{
Zustands\"ubergangsdiagramm der Befehle \Instr{LB[U]}, \Instr{LH[U]} und
\Instr{LW}. \(width\) ist je nach Befehl 1,~2~oder~4.
}

\Section{STORE}
Der Store-Befehl ist mit nur einer ALU und nur einer MMU nicht zu
parallelisieren, was dazu f\"uhrt, dass er der langsamste Befehl des Prozessors
ist. Da dieser Befehl jedoch generell auf RISC-Architekturen sehr langsam
ausgef\"uhrt wird, versuchen Compiler und Programmierer ohnehin
schreibende Speicherzugriffe zu vermeiden.

\Statemachine{./cu/img/store.pdf}{
Zustands\"ubergangsdiagramm des Befehle \Instr{SB}, \Instr{SH} und \Instr{SW}.
\(width\) ist je nach Befehl 1, 2 oder 4.
}

\Section{Timer und Counter}
Wie in der RISC-V-ISA gefordert gibt es einen Counter, der die Anzahl der
bisher ausgef\"uhrten Befehle speichert. Au\ss{}erdem gibt es einen Timer, der
die Anzahl der vergangenen Takte speichert. Da das FPGA keine Echtzeituhr
bereitstellt, wurde auch hierf\"ur der Taktz\"ahler verwendet.

Ausgelesen werden k\"onnen diese Counter durch die Befehle
\Instr{RDINSTRET[H]}\nolinebreak{}, \Instr{RDCYCLE[H]} und \Instr{RDTIME[H]}.

\Statemachine{./cu/img/timer_und_counter.pdf}{
Zustands\"ubergangsdiagramm der Befehle \Instr{RDINSTRET[H]},
\Instr{RDCYCLE[H]} und \Instr{RDTIME[H]}. \(counter\) sind die oberen bzw.
unteren 32 Bit des jeweiligen 64 Bit Z\"ahlers.
}
