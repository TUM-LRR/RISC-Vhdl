\Chapter{Das Leitwerk}
Das Leitwerk stellt die zentrale Steuereinheit des Prozessors dar. Es interpretiert
die Befehle und \"uberwacht ihre Ausf\"uhrung durch ALU und MMU. Dazu verwaltet
das Leitwerk den Program-Counter (PC) und das Instruction-Register (IR).

\Section{\"Uberblick}
Im Prozessor wurden die durch das \ISAname{RV32I Base Integer Instruction Set}
und die \ISAname{RV32M Standard Extension for Integer Multiplication and
Division} definierten Befehle implementiert. Insbesondere lag der Fokus dabei mehr auf Robustheit und
weniger auf Aspekten der Performanz.

Das Leitwerk besitzt f\"ur jeden Befehl eine eigene Zustandsmaschine. Welche dieser Zustandsmaschinen bei der Interpretation eines Befehls
ausgef\"uhrt wird, h\"angt dabei vom Inhalt des Instruction-Registers ab. Deshalb
muss jede m\"ogliche Zustandsabfolge zuletzt seinerseits das IR mit dem nachfolgenden Befehl belegen. Wann
dieser Befehl geladen wird kann nun innerhalb jedes Befehls einzeln implementiert und optimiert werden.

Um den Implementierungsaufwand bei \"Anderungen von Befehlen zu minimieren und gerade weil sich die Zustandsmaschinen im Einzelnen \"au\ss{}erst \"ahnlich sind,
wurde ein Compiler-Skript erstellt, das mehrere Makros bereitstellt, aus denen
dann die Befehlsimplementierungen zusammengebaut werden k\"onnen. Dieses Skript konvertiert dann
eine Abfolge, welche aus diesen Makros besteht, zu validem VHDL-Code.

\Subsection{Legende}
Da jeder Befehl wie bereits beschrieben durch eine eigene Zustandsmaschine ausgef\"uhrt wird, sind hier f\"ur jeden
Befehl ein separate Zustands\-\"uber\-gangs\-dia\-gramme gezeigt. Die Pr\"afixe
"`MMU:'' und "`ALU:'' werden genutzt, um anzuzeigen, dass eine Operation an die 
jeweiligen Einheit delegiert wird und das Leitwerk lediglich eine Anfrage an die entsprechende Komponente sendet.

Obwohl zwar in der der RISC-V ISA regelm\"a\ss{}ig verlangt wird, dass Operanden sign-extended werden, wurde dies in den Zustands\-\"uber\-gangs\-dia\-grammen der \"Ubersicht halber
ausgelassen. Die Implementierung allerdings umfasst diese Funktionalit\"aten selbstverst\"andlich wie in der ISA
beschrieben.

\Section{Integer Rechenbefehle}
Die Implementierung des Leitwerks wurde auf die in RV32I und RV32M definierten Rechenbefehle
optimiert, wodurch diese RISC-typischen Befehle sehr schnell
ausgef\"uhrt werden k\"onnen.

\Statemachine{./cu/img/integer_rechenbefehle.pdf}{
Zustands\"ubergangsdiagramm der Befehle \Instr{ADD[I]}, \Instr{SUB},
\Instr{SLT[I][U]}, \Instr{AND[I]}, \Instr{OR[I]}, \Instr{XOR[I]},
\Instr{SLL[I]}, \Instr{SRL[I]}, \Instr{SRA[I]}, \Instr{MUL[W]} und
\Instr{MULH[[S]U]}. \(op2\) ist entweder das Register, das mit \Ipart{rs2}
angegeben ist, oder eine Immediate (\Ipart{imm}). ``\(\circ\)'' repr\"asentiert die
jeweilige Operation (\(+\), \(-\), \dots{}).
}

\Subsection{Division und Modulo}
Da bei der Division unm\"oglich zu garantieren ist, dass diese immer innerhalb von drei
Takten erfolgreich beendet wird, muss das Leitwerk hier auf eine Best\"atigung der ALU
warten. Diese sieht vor, dass das Rechenwerk die Leitungen \Vhdl{alu\_data\_in}
auf 0 setzt.

\Statemachine{./cu/img/division.pdf}{
Zustands\"ubergangsdiagramm der Befehle \Instr{DIV[U][W]} und \Instr{REM[U][W]}.
}{``\(\circ\)'' repr\"asentiert die jeweilige Operation (\(/\), \(\bmod\),
\dots{}).}

\Section{LUI und AUPIC}
Da durch die Integer Rechenbefehle keine 32-Bit Immediates direkt geladen werden
k\"onnen, definiert die RISC-V-ISA die Befehle \Instr{LUI} und \Instr{AUPIC}, um
diesen Mangel beheben. Auch bei der Implementierung dieser Befehle wurde besonderer Fokus auf Performanz gelegt.

\Statemachine{./cu/img/lui_und_aupic.pdf}{
Zustands\"ubergangsdiagramm der Befehle \Instr{LUI} und \Instr{AUPIC}.
}{\(op2\) ist entweder 0 (bei \Instr{LUI}) oder der aktuelle Program-Counter
(bei \Instr{AUPIC}).}


\Section{Bedingte Spr\"unge}
Da bei Speicherzugriffen auf den DDR2-SDRAM Block der MMU\ref{ch:mmu}, dass ein Speicherzugriff ohne Zeit- und Datenverlust abgebrochen
werden kann, wurde auf eine einfache Branch-Prediction g\"anzlich verzichtet.
Dadurch geh\"ören die bedingten Spr\"unge im Hinblick auf Rechenzeit zu den teuersten Befehlen des
Prozessors.

\Statemachine{./cu/img/beq_und_bgeu.pdf}{
Zustands\"ubergangsdiagramm der Befehle \Instr{BEQ} und \Instr{BGE[U]}.
``\(\circ\)'' ist bei \Instr{BEQ} ``\(-\)'', bei \Instr{BGE} die SLT-Operation
und bei \Instr{BGEU} die SLTU-Operation.
}{}

\Statemachine{./cu/img/bne_und_bltu.pdf}{
Zustands\"ubergangsdiagramm der Befehle \Instr{BNE} und \Instr{BLT[U]}.             
``\(\circ\)'' ist bei \Instr{BNE} ``\(-\)'', bei \Instr{BLT} die SLT-Operation
und bei \Instr{BLTU} die SLTU-Operation.
}{}

\Section{Unbedingte Spr\"unge}
Anders als bei bedingten Spr\"ungen, kann bei unbedingten Spr\"ungen das
Zieladresse immer vorhergesagt werden, wodurch das Schreiben der R\"ucksprung-Adresse
und das Holen des n\"achsten Befehls parallel erfolgen kann, was letztlich zu einer
merklichen Geschwindigkeitssteigerung f\"uhrt.

\Statemachine{./cu/img/unbedingte_spruenge.pdf}{
Zustands\"ubergangsdiagramm der Befehle \Instr{JAL} und \Instr{JALR}. \(op2\) ist bei \Instr{JAL} der Program-Counter und bei \Instr{JALR} das Register,
welches durch \Ipart{rs1} adressiert wird.
}{}

\Section{LOAD}
Der LOAD-Befehl l\"adt aus dem Speicher immer einen 32-Bit Wert, den das
Leitwerk dann entsprechend beschneidet. Dies sollte urspr\"unglich die Implementierung der
MMU vereinfachen und aligned-Speicherzugriffe beschleunigen, wobei sich erst durch sp\"ater ver\"anderte Designentscheidungen ein nachteiliger Charakter in eben diesem Vorgehen offenbart hat. Die Ausf\"uhrungsdauer des Befehls f\"allt durchschnittlich hoch aus und wird in der
Praxis haupts\"achlich von der adressierten Speicherkomponente bestimmt.

\Statemachine{./cu/img/load.pdf}{
Zustands\"ubergangsdiagramm der Befehle \Instr{LB[U]}, \Instr{LH[U]} und
\Instr{LW}. \(width\) ist je nach Befehl 1,~2~oder~4.
}{}

\Section{STORE}
Der Store-Befehl ist mit nur einem Rechen- und Speicherwerk nicht zu
parallelisieren, was dazu f\"uhrt, dass er den langsamsten Befehl des Prozessors
darstellt. Da Schreibzugriffe jedoch generell auf RISC-Architekturen traditionell sehr hohe Kosten hinsichtlich der ben\"otigten Rechenzeit aufweisen, sind Compiler und Programmierer ohnehin dazu angehalten, schreibende Speicherzugriffe wo m\"oglich zu vermeiden.

\Statemachine{./cu/img/store.pdf}{
Zustands\"ubergangsdiagramm des Befehle \Instr{SB}, \Instr{SH} und \Instr{SW}.
\(width\) ist je nach Befehl 1, 2 oder 4.
}{}

\Section{Timer und Counter}
Wie in der RISC-V-ISA gefordert existiert des Weiteren ein Counter, der die Anzahl der bisher ausgef\"uhrten Befehle z\"ahlt. Au\ss{}erdem umfasst die Implementierung auch einen Timer, der analog dazu die Anzahl der bereits vergangenen Takte speichert. Da das Entwicklungsboard allerdings keine Echtzeituhr
bereitstellt, greift die Implementierung dieser Funktionalit\"at auf den Taktz\"ahler zur\"uck.

Ausgelesen werden k\"onnen diese Counter durch die Befehle
\Instr{RDINSTRET[H]}\nolinebreak{}, \Instr{RDCYCLE[H]} und \Instr{RDTIME[H]}.

\Statemachine{./cu/img/timer_und_counter.pdf}{
Zustands\"ubergangsdiagramm der Befehle \Instr{RDINSTRET[H]},
\Instr{RDCYCLE[H]} und \Instr{RDTIME[H]}. \(counter\) sind die oberen bzw.
unteren 32 Bit des jeweiligen 64 Bit Z\"ahlers.
}{}
