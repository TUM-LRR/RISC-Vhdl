\Chapter{Die Arithmetische Logische Einheit}
Die Arithmetisch Logische Einheit ist verantwortlich f\"ur die Durchf\"uhrung aller f\"ur die Befehlsausf\"uhrung durch das Leitwerk relevanten Rechenoperationen. Zusätzlich verwaltet sie die Register des Prozessors.

\Section{\"Uberblick}
Zentrale Design-Idee hinter der ALU ist es, mehr als einen reinen Multiplexer f\"ur Befehle zu entwickeln. Stattdessen bietet sie dem Leitwerk ein Interface,
\"uber das Operationen auf Registern oder Immediates angefragt werden k\"onnen, wozu eine Menge von Prozessor-internen OPCodes definiert wurden (SIEHE …).

Da die ALU auch in nicht-arithmetischen Maschinenbefehlen, wie Spr\"ungen, oft gebraucht wird, sollte der Synchronisations- und Kommunikationsaufwand zwischen ALU und Leitwerk m\"oglichst reduziert werden.
Aus diesem Grund werden grundsätzlich alle Befehle (exklusive Division) mittels einer State-Machine innerhalb von drei Takten ausgef\"uhrt (SIEHE …). Aus diesem Grund ist kein Synchronisations-Protokoll zwischen ALU und Leitwerk von N\"oten.

Um die Komplexität des Leitwerks zu reduzieren, wurde die Low-Level-Verwaltung der Register in die ALU ausgelagert. Diese wurden als Blockram realisiert, um die Anzahl an belegten Slices auf dem FPGA zu reduzieren.

\Section{Das Interface}
HIER KOMMT EIN BILD VOM INTERFACE ODER S\"O

Neben zwei 32-Bit-f\"ur Operanden, gibt es einen dedizierten Eingang f\"ur Registeradressen sowie einen zur Auswahl der gew\"unschten Operation. 
Der Adresseingang dient hierbei lediglich zur Adressierung des Zielregisters. Da alle Befehle nur aus zwei Operanden bestehen, werden die Operanden-Eingänge entweder zur \"Ubermittlung von Immediates, oder zur Adressierung der Operanden-Register verwendet, um die Anzahl der Eingänge zu reduzieren.

Aktiviert wird die ALU vom Leitwerk \"uber \Vhdl{Cu\_work\_in}, innerhalb von drei Takten liegt dann auf \Vhdl{Cu\_data\_out} das entsprechende Ergebnis an.

\Section{Interne Befehle}
Zur Auswahl der gew\"unschten Operation bietet die ALU einen 7 Bit breiten Befehls-Eingang an.
Jeder Befehle besteht dabei aus drei Sektionen\vspace{10}:

\Instr{Reg1|Reg2|OpC}\vspace{5}

Das oberste Bit \Ipart{Reg1} entscheidet dar\"uber, ob der erste Operand als Immediate vom Dateneingang \Vhdl{ Cu\_data\_in1} genommen, oder aus dem durch \Vhdl{Cu\_data\_in1} adressierten Register.
Analog entscheidet \Ipart{Reg2} dar\"uber, ob auf \Vhdl{Cu\_Data\_in2} eine Adresse oder ein Immediate anliegt.
Darauf folgt der f\"unf Bit umfassende interne OpCode. Zur Verf\"ugung stehen OpCodes f\"ur:

\begin{itemize}
\item Addition, Subtraktion
\item Logische Shifts
\item Arithmetische Shifts
\item Set Less Than Immediate
\item Multiply Lower auf Signed und Unsigned-Operanden
\item Multiply Upper auf Signed und Unsigned Operanden
\item Division Signed und Unsigned
\item Modulo-Rechnung Signed und Unsigned
\end{itemize}

\Section{Befehlsausf\"uhrung}
Die Ausf\"uhrung von Befehlen ist \"uber eine State-Machine mit drei zentralen Zuständen, sowie zwei Zusatzzuständen f\"ur Signed- und Unsigned-Division bzw. Modulo-Operationen. In jedem der drei zentralen Zustände wird dabei jeweils nur f\"ur einen Takt verblieben.

\Subsection{State 1 - Selektion der Operanden}
Die ALU verbleibt in State 1, bis vom Leitwerk das entsprechende Work-Signal gesendet wird.
Anschließend werden auf Grundlage des IN SEKTION (…) BESCHRIEBENEN Befehls zwei Operanden-Signale \Vhdl{s\_op1} und \Vhdl{s\_op2} mit einem Immediate vom Daten-Eingang belegt, oder es wird ein Register-Lesezugriff angestoßen, dessen Ergebnis im nächsten Takt zur Verf\"ugung steht.

\Subsection{State 2 - Operatonsausf\"uhrung}
State 2 dient der eigentlichen Befehlsausf\"uhrung. Realisiert ist er als Case-Statement \"uber den internen OpCode. Innerhalb jedes Cases wird zwischen den vier m\"oglichen Kombinationen auf Immediate- und Register-Operanden unterschieden.
Dies ist n\"otig, da das Ergebnis eines m\"oglichen Register-Zugriffs erst in diesem Takt anlegt und deshalb nicht einfach die \Vhdl{s\_op1} und \Vhdl{s\_op2}-Signale f\"ur Immediate-Operationen \"uberschrieben werden k\"onnen.
Die Operation wird auf den jeweiligen Operanden durchgef\"uhrt und in einem Akkumulator gespeichert.\vspace{10}

Ein Großteil der Operationen ist \"uber die \Vhdl{IEEE.NUMERIC\_STD.ALL} Operatoren realisiert und innerhalb dieses Taktes vollständig abgeschlossen.
Da die Standard-Multiplikation von 32-Bit-Werten zu einem 64 Bit langen Ergebnis f\"uhrt, werden Multiplikationsergebnisse nicht im normalen Akkumulator-Signal \Vhdl{acc}, sondern im Zusatzsignal \Vhdl{Mult-Result} gespeichert. Der VHDL-Compiler erlaubte es nicht, unmittelbar auf das Ergebnis zuzugreifen und entweder die oberen oder unteren 32 Bit in \Vhdl{acc} zu speichern.\vspace{10}

Da der Standard-Operator \Vhdl{sar} nicht durch die gegebenen Entwicklungsumgebung synthetisierbar ist, wird bei einem arithmetischen Rechtsshift um {n} Stellen zusätzlich das oberste Bit des ersten Operanden zwischengespeichert, um im Nachfolgenden State – wenn n\"otig – die oberen {n} Bits auf {1} zu setzen.\vspace{10}

Im Falle einer Modulo- oder Unsigned-Division wird in State 2 der \"Ubergang in State 4 bzw. 5 eingeleitet. Ansonsten erfolgt ein direkter \"Ubergang in State 3.

\Subsection{State 3 - Write-Back}
In State 3 werden die akkumulierten Ergebnisse in das von der ALU adressierte Register geschrieben.
Im Großteil der Operationen erfolgt dies unmittelbar. Bei Multiplikationsoperationen muss jedoch zuerst an Hand des OpCodes entschiedene werden, ob die oberen oder unteren 32 Bit des Ergebnisses gespeichert werden sollen. Abhängig vom Status-Bit f\"ur arithmetische Shifts um {n} Stellenwerden, wie oben beschrieben, wenn n\"otig die oberen n Bits des Ergebnisses auf {1} gesetzt.\vspace{10}

Neben dem Speichern des Ergebnisses werden zwei zusätzliche Ausgänge belegt:

Auf \Vhdl{cu\_Data\_out} wird das Ergebniss angelegt. Einzige Ausnahme stellt die Division bzw. Modulorechnung dar, bei der als Synchronisationssignal der Daten-Ausgang genullt wird. Das ist n\"otig, da die Division vom normalen 3-Takte-Schema abweicht.
Auch die Debug-Schnittstelle erhält \"uber \Vhdl{debug\_signal} den entsprechenden Wert sowie das relevante Register \"uber \Vhdl{debug\_adr\_signal}.

\Subsection{State 4 - Division Unsigned}
Um die Anzahl an n\"otigen Takten zu reduzieren, wird bei der Umsetzung der Unsigned Division eine durch den Xilinx-Core-Generator erstellte Divisionseinheit genutzt, welche eine gepipelinte Variante der SRT-Division durchf\"uhrt. Obwohl der Prozessor keinen unmittelbaren Nutzen aus dem Pipelining zieht, kann durch die effiziente Implementierung die Anzahl an n\"otigen Takten reduziert werden. Zusätzlich liefert die Einheit sowohl den Rest, als auch das Divisionsergebnis, weshalb die beiden Operationen gleich behandelt werden k\"onnen.

Hierzu werden bei Betreten des States die Operanden angelegt. Anschließend wird mit einem Zähler gewartet, bis das Ergebnis der Operation anliegt.
Anschließend werden entweder der Rest oder das Ergebnis in den Akkumulator eingelesen und in State 3 \"ubergegangen.

\Subsection{State 5 - Division Signed}
Die Umsetzung ist analog zu State 4, abgesehen davon, dass eine Einheit zur Durchf\"uhrung von Signed-Divisionen verwendet wird.

\Section{Die Register}
Insgesamt stellt die ALU 32 Register mit 32 Bit Breite bereit, wobei Register 0 konstant den Wert 0 liefert und nicht \"uberschreibbar ist. Dies ist n\"utzlich zum unveränderten Laden eines Registers.

Implementiert wurden die Register als Dual-Port-Block-Ram. Dies erm\"oglicht den zeitgleichen Zugriff auf zwei verschiedene Speicherinhalte, was bei Register-Register-Operationen vorkommt. Zusätzlich wird dadurch die Anzahl an verwendeten Slices reduziert, da dedizierte BlockRam-Bausteine logisch zum Registersatz zusammengef\"ugt werden.\vspace{10}

Die Umsetzung erfolgte dabei nicht \"uber eine Core-Generierte Variante, sondern durch die Einhaltung eines speziellen Verwendungsprotokolls, sodass der Compiler das definierte std_logic_vector-Array automatisch in einen BlockRam umsetzt.

Voraussetzung daf\"ur ist, dass nicht direkt auf Register-Inhalten operiert wird, sondern sie zuerst in einem Signal zwischengespeichert und im nächsten Takt verwendet werden. Dies garantiert die 3-stufige State-Machine der ALU.

\Section{Reset}
Die ALU verf\"ugt \"uber einen Reset-Eingang, welcher direkt vom CPU-Toplevel zu ihr durchgeleitet wird. Bei einem Reset wird sie in den State 0 \"uberf\"uhrt, um nach Ende des Resets Befehle des Leitwerks entgegennehmen zu k\"onnen.

Zusätzlich wird der Divisions-Flankenzähler genullt, um keine Fehler bei Nachfolgenden Operationen zu versursachen. Das Reset-Signal wird auch an den Reset (SCLR)-Eingang der Divisionseinheit durchgeleitet.

Ein Reset der Register erfolgt nicht, abgesehen vom Register 0 ist f\"ur den Entwickler eines Nutzerprogramms keine Aussage \"uber die enthaltenen Werte der Register m\"oglich.

